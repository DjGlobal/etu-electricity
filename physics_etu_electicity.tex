\documentclass[a4paper,10pt]{book}
\usepackage[utf8]{inputenc}
\usepackage[T2A]{fontenc}
\usepackage[warn]{mathtext}
\usepackage[russian]{babel}
\usepackage{amsmath, amsfonts, amssymb}
\let\vec\mathbf

\begin{document}
% Раздел 2
% Лекция 13
\part{Раздел II}
\chapter{Лекция 13}
\section*{Понятие о токе}
В предыдущем разделе рассматривались неподвижные электрические заряды и создаваемое ими электрическое поле. Перейдём 
к изучению движущихся зарядов. Электрическим током называют направленное перемещение заряженных частиц или движение 
заряженных тел. Ток, образуемый движением заряженных микрочастиц в твердых, жидких или газообразных телах под действием
электрического поля называют \emph{током проводимости}. Заряженные частицы, движение которых образует электрический ток,
получили название \emph{носителей тока}.

В металлах ток создается движние свободных электронов. В жидкостях носителями тока служат положительные и отрицательные 
ионы (ионный ток). В газах ток создается движением положительных и отрицательных ионов, а также электронов. Носителями 
тока в полупроводниках являются электроны и ``дырки'' (свободные места, на которые могут переходить электроны), частично 
подобные положительные положительным зарядам.

Кроме тока проводимости различают еще \emph{ток в вакуума}, например поток электронов в электронной лампе, телевизионной 
трубке. 

Движение заряженных тел (макроскопических) называют \emph{конвекционным (переносным) током}.

При движении любой заряженной частицы или тела в окружающем пространстве образуется магнитное поле. Поэтому основным свойством 
всякого тока - проводимости, в вакууме и конвекционного - является образование магнитного поля (магнитное действие тока). Прох
ождение тока в твердых\footnote{За исключением сверхпроводников}, жидких и газообразных телах сопровождается их нагреванием в 
результате частичного превращения упорядоченного действия тока в хаотическое (тепловое действие тока).

Во всех случаях ионного тока наблюдается перенос вещества и некоторые химические процессы (химическое действие тока).
\section{Сила тока}
Силой тока \emph{(i, I)} называют скалярную величину равную отношению количества электричества \emph{dq}, проходящего через некоторую
поверхность, ко времени прохождения \emph{dt}. Обычно этой поверхностью служит поперечное сечение проводника.
\begin{equation}\label{iless}
 i = \frac{\mathrm{d}q}{\mathrm{d}t}
\end{equation}
Формула \ref{iless} пригодна и для меняющегося и для постоянного тока. Если же сила тока неизменна во времени, то величина её \emph{I}
определится как отношение \emph{q} к \emph{t}.
\begin{equation}\label{igreat}
 I = \frac{q}{t}
\end{equation}
Иногда вместо термина ``сила тока'' говорят просто ``ток''.

Единицей силы тока служит ампер (\emph{a}). Определение этой основной единицы Международной системы приведено в разделе ``Электромагнетизм''
(лекция 34).

Из формулы \ref{igreat} можно определить производную единицу заряда в СИ
\begin{equation}
 1 \text{к} = 1 \text{а} \cdot 1 \text{сек} \nonumber
\end{equation}
Один кулон - это заряд, который проходит через сечение проводинка за 1 \emph{сек} при силе тока в 1 \emph{а}. При этом через сечение проводника
пройдёт 
\begin{equation}
 N = \frac{1}{e} = \frac{1 \text{к}}{1,6 \cdot 10^-19 \text{к}} = 6,25 \cdot 10^18 \nonumber
\end{equation}
элементарных зарядов.

Формула \ref{igreat} используется в системе СГС для определения единицы силы тока
\begin{equation}\label{sgsamp}
  1\text{СГС}_I = \frac{1\text{СГС}_q}{1 \text{cек}} = \frac{1}{3 \cdot 10^9} \text{а}
\end{equation}
\section{Плотность тока}
В учении о токе важную роль играет векторная величина плотность тока \textbf{j}, численно равная силе тока, отнесенное к единице площади
поперечного сечения проводника с током.
\begin{equation}
 j = \frac{I}{S}
\end{equation}
Плотность тока измеряется в $\text{а/м}^2$ (внесистемная единица $1\frac{\text{а}}{\text{мм}^2} = 10^6\frac{\text{а}}{\text{м}^2}$).

Если рассмотреть проводник с переменным сечением или проводящую среду, то плотность тока \emph{j} будет величиной переменной
\begin{equation}\label{dencity}
 j = \frac{\mathbf{d}I}{\mathbf{d}S_n}
\end{equation}
где $dS_n$ - перпендикулярный к направлению плотности тока элемент площади.

Тогда
\begin{equation}\label{inti}
 I = \int\limits_{s}j\mathbf{d}S_n
\end{equation}
т.е. сила тока является потоком от вектора плотности тока через заданную поверхность (см. лекцию 2).
\section{Закон Ома}
В 1827 г. немецкий учитель физики Ом установил опытным путём пропорциональность между напряжением \emph{U}, приложенным 
к участку цепи, и током, созданным в нём. Отношение напряжения (разности потенциалов) к силе тока в данном участке цепи есть величина
постоянная называемая сопротивлением участка
\begin{equation}\label{resist}
 \frac{U}{I} = R
\end{equation}
Как известно, сопротивление проводника \emph{R} постоянного сечения связанно с его длиной \emph{L}, площадью поперечного сечения \emph{S}
и удельным сопротивлением $\rho$ следующим соотношением:
\begin{equation}\label{rls}
 R = \rho\frac{L}{S} = \frac{1}{\gamma}\frac{L}{S}
\end{equation}
Величина, обратная сопротивлению, - $G = \frac{1}{R}$ называется проводимостью, а $\gamma = \frac{1}{\rho}$ - удельной проводимостью
(электропроводимостью).

Из формулы \ref{resist} устанавливается единица для измерения сопротивления в СИ
\begin{equation}
 1 \text{ом} = \frac{1\text{в}}{1\text{a}},
\end{equation}
т.е. 1 \emph{ом} является сопротивлением проводника, в котором идет ток в 1 \emph{а} при напряжении 1\emph{в} между его концами.

Удельное сопротивление $\rho$ численно равно сопротивлению проводника из данного материала, длиной в 1 \emph{м} и поперечным сечением
в 1 $\text{м}^2$ (в СИ); измеряется величина $\rho$ в $\frac{\text{ом} \cdot \text{м}^2}{\text{м}} = \text{ом} \cdot \text{м}$.
Внесистемная единица $1 \text{ом} \cdot \text{см} = 10^-2 \text{ом} \cdot \text{м}$.

Как известно из электростатики, напряженность однородного поля численно равна падению потенциала на единицу расстояния
\begin{equation}\label{Ell}
 E = \frac{U}{L}
\end{equation}
Используя формулы \ref{igreat} \ref{dencity} \ref{resist} \ref{rls} и \ref{Ell}, получаем следующее выражение для плотности тока:
\begin{equation}\label{longf}
 j = \frac{I}{S} = \frac{U}{RS} = \frac{U}{\rho\frac{L}{S}S} = \frac{1}{\rho}\frac{U}{L} = \gamma E
\end{equation}
где \emph{E} - напряженность электрического поля, созданного в проводнике.

Так как \textbf{E} - величина векторная, то
\begin{equation}\label{vecj}
 \vec{j} = \gamma\vec{E}
\end{equation}
т.е. вектор \textbf{j} совпадает по направлению с вектором напряженности электрического поля.

Формула \ref{vecj} выражает закон Ома в \emph{дифференциальной форме}. Величина силы тока получается в общем случае путем интегрирования
плотности тока по площади [см. формулу \ref{inti}]. Плотность тока оказывается прямопопорциональной напряженности поля, созданного в проводнике.
\section{Работа и мощность тока. Закон Джоуля-Ленца в дифференциальной форме}
При прохождении тока по проводнику совершается работа по перенесению заряда q между точками с разностью потенциалов \emph{U}. Эта работа 
затрачивается на нагревание проводника
\begin{equation}\label{work}
 A = qU = IUt = I^2Rt = \frac{U^2}{R}t = Q
\end{equation}
В Международной системе единиц работа \emph{A} и количество тепла \emph{Q} измеряются в джоулях. Поэтому формула \ref{work} для работы 
тока одновременно выражает закон Джоуля-Ленца о тепловом действии тока. 

Заметим, что при \emph{последовательном} соединении сила тока \emph{I} во всех сечениях одинакова и поэтому, согласно формуле $Q = I^2Rt$,
количество тепла , выделяющееся на каком-либо участке цепи, пропорционально его сопротивлению $R$.

При \emph{параллельном} соединении во всех ветвях имеется общее напряжение U и количество тепла
\begin{equation}
 Q = \frac{U^2}{R}t, \nonumber
\end{equation}
выделяющееся в какой-либо ветви, обратно пропорционально ее сопротивлению, т.е. пропорционально ее проводимости.

Подсчитаем количество тепла, выделяемое за единицу времени в единице объема $V$ проводника, т.е. найдём 
\begin{equation}
 Q_1 = \frac{Q}{tV} = \frac{Q}{tSL}. \nonumber
\end{equation}
Используя соотношения \ref{rls}, \ref{Ell} и \ref{work}, получаем:
\begin{equation}\label{warm}
 Q_1 = \frac{U^2}{RSL} = \frac{E^2L^2}{\frac{1}{\gamma}\frac{L}{S}SL} = \gamma E^2.
\end{equation}
\emph{Колличество тепла, выделяемое в единице объема проводника за единицу времени, пропорционально квадрату напряженности электрического 
поля созданного в проводнике.}

Соотношение \ref{warm} называется законом Джоуля-Ленца в дифференциальной форме, так как определяет количество тепла, выделяемое в 1 сек
в единице объема.

В общем случае количнство тепла, выделяемое в 1 сек во всем объеме,
\begin{equation}\label{warmVol}
 Q_V = \int\limits_{(V)}Q_1\mathbf{d}V
\end{equation}
Рассматривая формулы \ref{vecj} и \ref{warm}, замечаем, какую важную роль играет величина напряженности электрического поля $E$, созданного
источником в проводнике. Величина $E$ определяет плотность тока в цепи и количество выделяемого током тепла.
\chapter{Лекция 14}
\section*{Замкнутая цепь с источником тока}
Рассмотрим замкнутую цепь электрического тока, составленную из источника тока и внешнего сопротивления $R$ \ref{img1}. Во внешней части цепи
заряды движутся под действием электростатических сил от большего потенциала к меньшему (путь $1R2$). Для того чтобы заставить заряды
двигаться внутри источника тока против направления электростатического поля (от 2 к 1 через внутренней сопротивление $r$), необходимо наличие
сил неэлектрического происхождения, так называемых сторонних сил.

Величину, определяемую работой, которую совершают сторонние силы при перемещении единичного положительного заряда по всей замкнутой цепи,
называют электродвижущей силой (э. д. с.) $\mathcal{E}$ в этой цепи (не путать с напряженностью электрического поля $\mathbf{E}$ и ее 
числовым значением $E$).
\begin{equation}\label{eds}
 \mathcal{E} = \frac{A_\text{ст}}{q_0}
\end{equation}
% here img1
Этой работой измеряется электростатическая энергия, которая получается в источнике тока за счет иных, неэлектрических форм энергии.

Из определения ЭДС видно, что это величина, аналогичная разности потенциалов (напряжению), имещая ту же размерность и измеряемая в тех же
единицах - вольтах.

Падение напряжения во внешней части цепи $U_1$ измеряется работой, совершаемой при перемещении единичного положительного заряда через 
внешнее сопротивление $R$
\begin{equation}\label{u1}
 U_1 = \frac{A_1}{q_0}
\end{equation}
При прохождении тока по всей замкнутой цепи внутри источника также затрачивается энергия на продвижение зарядов и имеет место внутреннее
падение напряжения $U_2$ эта энергия превращается в джоулево тепло внутри источника.
\begin{equation}\label{u2}
 U_2 = \frac{A_2}{q_0}
\end{equation}
Учитывая, что все три величины - $\mathcal{E}, U_1, \text{и} U_2$ измеряются работой по перемещению единичного положительного заряда и 
применяя закон сохранения энергии, получим
\begin{equation}\label{Aex}
 A_\text{ст} = A_1 + A_2
\end{equation}
\begin{equation}\label{Aex/q0}
 \frac{A_\text{ст}}{q_0} = \frac{A_1}{q_0} + \frac{A_2}{q_0} \text{ и } \mathcal{E} = U_1 + U_2.
\end{equation}
Эдс равна сумме падений напряжения во внешней и внутренней частях замкнутой цепи.

Устройства, в которых сторонние силы совершают работы и происходит превращение какого-либо вида энергии в электрическую, называются генераторами,
или источниками эдс (реже - источниками тока)ю

Подавляющая часть электрической энергии совершается за счет совершения механической работы в машинах, где используется явление электромагнитной
индукции (генераторы на стационарных и передвижных электростанциях). В термоэлементах (термопарах) происходит превращение тепловой энергии 
в электрическую. В гальванических элементах и аккумуляторах химическая энергия преобразуется в электрическую. Некоторые виды фотоэлементов
позволяют получать эдс за счет лучистой энергии.

Таким образом, следует различать электромеханические, тепловые, химические и оптические генераторы, или источники эдс. В настоящее время
разрабатываются новые генераторы - магнитогидродинаические, где тепловая энергия нагретого ионизированного газа или дыма непосредственно 
превращается в электрическую.

Получение различных радиоактивных изотопов позволило создать маломощные генераторы длительного действия; в них эдс получается за счет непрерывного
выбрасывания электронов радиоактивными ядрами.
\section{Закон Ома для всей цепи}
Используя закон Ома для участка цепи, можно заменить в формуле \ref{Aex/q0} $U_1$ и $U_2$ следующими выражениями:
\begin{equation}
 U_1 = IR,\text{ }U_2 = Ir.\nonumber
\end{equation}
Тогда соотношение \ref{Aex/q0} превратится в закон Ома для всей цепи: 
\begin{equation}\label{full_Ohm}
 \frac{\mathcal{E}}{I} = R + r
\end{equation}
Отношение эдс, действующей в замкнутой цепи, к силе тока есть величина постоянная, равная полному сопротивлению цепи или 
\begin{equation}\label{full_I}
 I = \frac{\mathcal{E}}{R + r}.
\end{equation}
Максимальный ток $I_\text{к.з.}$, как следует из закона Ома, получится если внешнее сопротивление R = 0, т.е. имеет место короткое замыкание
\begin{equation}\label{I_shcut}
 I_\text{к.з.} = \frac{\mathcal{E}}{r}
\end{equation}
Записав формулу \ref{full_Ohm} в виде
\begin{equation}
 \mathcal{E} = U_1 + Ir\nonumber
\end{equation}
можно определить эдс как величину, численно равную напряжению на зажимах источника при $I = 0$, т.е. при разомкнутой внешней цепи.
Этим пользуются для практического измерения эдс. Точное измерение эдс производится компенсационным способом (известным из лабораторной 
работы), при котором ток от источника не потребляется. Приближенно эдс измеряется вольтметром с большим сопротивлением.
\section{Полная и полезная мощноть. КПД источника}
Мощность, выделяемая во внешней цепи (нагрузке $R$), называется полезной
\begin{equation}
 P_1 = U_1I = I^2R\nonumber
\end{equation}
Внутри источника тока расходуется мощность 
\begin{equation}
 P_2 = I^2r\nonumber
\end{equation}
Полная мощность замкнутой цепи
\begin{equation}\label{full_P}
 P_n = I^2R + I^2r = IU_1 + IU_2 = I\mathcal{E}.
\end{equation}
Найдем условие, при котором полезная мощность будет максимальной. Для этого выразим полезную мощность как 
\begin{equation}
 P_1 = P_n - P_2 = I\mathcal{E} - I^2r\nonumber
\end{equation}
и приравняем нулю производную
\begin{equation}
 \frac{\mathbf{d}P_1}{\mathbf{d}I} = \mathcal{E} - 2Ir = 0,\nonumber
\end{equation}
откуда
\begin{equation}\label{profit_I}
 I = \frac{\mathcal{E}}{2r}.
\end{equation}
Сравнивая эту формулу с законом Ома \ref{full_I}, видим, что максимальна мощность во внешней цепи выделяется при равенстве внешнего и 
внутреннего сопротивления:
\begin{equation}\label{exeqin}
 R = r.
\end{equation}
Коэффициентом полезного действия $\eta$ источника эдс называется отношение полезной мощности к полной
\begin{equation}\label{nu}
 \eta = \frac{P_1}{P_n} = \frac{I^2R}{I^2(R + r)} = \frac{U_1}{\mathcal{E}} = \frac{R}{R + r}
\end{equation}
Наибольший кпд, равный единице, получится при равенстве нулю внутреннего сопротивления источника. Исходя из этого стараются делать внутреннее
сопротивление источника эдс (обмоток механического генератора, электролита и пластин аккумулятора) минимальным.

















\end{document}
