\documentclass[a4paper,10pt]{book}
\usepackage[utf8]{inputenc}
\usepackage[T2A]{fontenc}
\usepackage[warn]{mathtext}
\usepackage[russian]{babel}
\usepackage{amsmath, amsfonts, amssymb}
\let\vec\mathbf
\setcounter{chapter}{12}

\begin{document}
% Раздел 2
% Лекция 13
\part{Раздел II}
\chapter{Лекция 13}
\section*{Понятие о токе}
В предыдущем разделе рассматривались неподвижные электрические заряды и создаваемое ими электрическое поле. Перейдём 
к изучению движущихся зарядов. Электрическим током называют направленное перемещение заряженных частиц или движение 
заряженных тел. Ток, образуемый движением заряженных микрочастиц в твердых, жидких или газообразных телах под действием
электрического поля называют \emph{током проводимости}. Заряженные частицы, движение которых образует электрический ток,
получили название \emph{носителей тока}.

В металлах ток создается движние свободных электронов. В жидкостях носителями тока служат положительные и отрицательные 
ионы (ионный ток). В газах ток создается движением положительных и отрицательных ионов, а также электронов. Носителями 
тока в полупроводниках являются электроны и ``дырки'' (свободные места, на которые могут переходить электроны), частично 
подобные положительные положительным зарядам.

Кроме тока проводимости различают еще \emph{ток в вакуума}, например поток электронов в электронной лампе, телевизионной 
трубке. 

Движение заряженных тел (макроскопических) называют \emph{конвекционным (переносным) током}.

При движении любой заряженной частицы или тела в окружающем пространстве образуется магнитное поле. Поэтому основным свойством 
всякого тока - проводимости, в вакууме и конвекционного - является образование магнитного поля (магнитное действие тока). Прох
ождение тока в твердых\footnote{За исключением сверхпроводников}, жидких и газообразных телах сопровождается их нагреванием в 
результате частичного превращения упорядоченного действия тока в хаотическое (тепловое действие тока).

Во всех случаях ионного тока наблюдается перенос вещества и некоторые химические процессы (химическое действие тока).
\section{Сила тока}
Силой тока \emph{(i, I)} называют скалярную величину равную отношению количества электричества \emph{dq}, проходящего через некоторую
поверхность, ко времени прохождения \emph{dt}. Обычно этой поверхностью служит поперечное сечение проводника.
\begin{equation}\label{iless}
 i = \frac{\mathrm{d}q}{\mathrm{d}t}
\end{equation}
Формула \ref{iless} пригодна и для меняющегося и для постоянного тока. Если же сила тока неизменна во времени, то величина её \emph{I}
определится как отношение \emph{q} к \emph{t}.
\begin{equation}\label{igreat}
 I = \frac{q}{t}
\end{equation}
Иногда вместо термина ``сила тока'' говорят просто ``ток''.

Единицей силы тока служит ампер (\emph{a}). Определение этой основной единицы Международной системы приведено в разделе ``Электромагнетизм''
(лекция 34).

Из формулы \ref{igreat} можно определить производную единицу заряда в СИ
\begin{equation}
 1 \text{к} = 1 \text{а} \cdot 1 \text{сек} \nonumber
\end{equation}
Один кулон - это заряд, который проходит через сечение проводинка за 1 \emph{сек} при силе тока в 1 \emph{а}. При этом через сечение проводника
пройдёт 
\begin{equation}
 N = \frac{1}{e} = \frac{1 \text{к}}{1,6 \cdot 10^-19 \text{к}} = 6,25 \cdot 10^18 \nonumber
\end{equation}
элементарных зарядов.

Формула \ref{igreat} используется в системе СГС для определения единицы силы тока
\begin{equation}\label{sgsamp}
  1\text{СГС}_I = \frac{1\text{СГС}_q}{1 \text{cек}} = \frac{1}{3 \cdot 10^9} \text{а}
\end{equation}
\section{Плотность тока}
В учении о токе важную роль играет векторная величина плотность тока \textbf{j}, численно равная силе тока, отнесенное к единице площади
поперечного сечения проводника с током.
\begin{equation}\label{ji}
 j = \frac{I}{S}
\end{equation}
Плотность тока измеряется в $\text{а/м}^2$ (внесистемная единица $1\frac{\text{а}}{\text{мм}^2} = 10^6\frac{\text{а}}{\text{м}^2}$).

Если рассмотреть проводник с переменным сечением или проводящую среду, то плотность тока \emph{j} будет величиной переменной
\begin{equation}\label{dencity}
 j = \frac{\mathbf{d}I}{\mathbf{d}S_n}
\end{equation}
где $dS_n$ - перпендикулярный к направлению плотности тока элемент площади.

Тогда
\begin{equation}\label{inti}
 I = \int\limits_{s}j\mathbf{d}S_n
\end{equation}
т.е. сила тока является потоком от вектора плотности тока через заданную поверхность (см. лекцию 2).
\section{Закон Ома}
В 1827 г. немецкий учитель физики Ом установил опытным путём пропорциональность между напряжением \emph{U}, приложенным 
к участку цепи, и током, созданным в нём. Отношение напряжения (разности потенциалов) к силе тока в данном участке цепи есть величина
постоянная называемая сопротивлением участка
\begin{equation}\label{resist}
 \frac{U}{I} = R
\end{equation}
Как известно, сопротивление проводника \emph{R} постоянного сечения связанно с его длиной \emph{L}, площадью поперечного сечения \emph{S}
и удельным сопротивлением $\rho$ следующим соотношением:
\begin{equation}\label{rls}
 R = \rho\frac{L}{S} = \frac{1}{\gamma}\frac{L}{S}
\end{equation}
Величина, обратная сопротивлению, - $G = \frac{1}{R}$ называется проводимостью, а $\gamma = \frac{1}{\rho}$ - удельной проводимостью
(электропроводимостью).

Из формулы \ref{resist} устанавливается единица для измерения сопротивления в СИ
\begin{equation}
 1 \text{ом} = \frac{1\text{в}}{1\text{a}},
\end{equation}
т.е. 1 \emph{ом} является сопротивлением проводника, в котором идет ток в 1 \emph{а} при напряжении 1\emph{в} между его концами.

Удельное сопротивление $\rho$ численно равно сопротивлению проводника из данного материала, длиной в 1 \emph{м} и поперечным сечением
в 1 $\text{м}^2$ (в СИ); измеряется величина $\rho$ в $\frac{\text{ом} \cdot \text{м}^2}{\text{м}} = \text{ом} \cdot \text{м}$.
Внесистемная единица $1 \text{ом} \cdot \text{см} = 10^-2 \text{ом} \cdot \text{м}$.

Как известно из электростатики, напряженность однородного поля численно равна падению потенциала на единицу расстояния
\begin{equation}\label{Ell}
 E = \frac{U}{L}
\end{equation}
Используя формулы \ref{igreat} \ref{dencity} \ref{resist} \ref{rls} и \ref{Ell}, получаем следующее выражение для плотности тока:
\begin{equation}\label{longf}
 j = \frac{I}{S} = \frac{U}{RS} = \frac{U}{\rho\frac{L}{S}S} = \frac{1}{\rho}\frac{U}{L} = \gamma E
\end{equation}
где \emph{E} - напряженность электрического поля, созданного в проводнике.

Так как \textbf{E} - величина векторная, то
\begin{equation}\label{vecj}
 \vec{j} = \gamma\vec{E}
\end{equation}
т.е. вектор \textbf{j} совпадает по направлению с вектором напряженности электрического поля.

Формула \ref{vecj} выражает закон Ома в \emph{дифференциальной форме}. Величина силы тока получается в общем случае путем интегрирования
плотности тока по площади [см. формулу \ref{inti}]. Плотность тока оказывается прямопопорциональной напряженности поля, созданного в проводнике.
\section{Работа и мощность тока. Закон Джоуля-Ленца в дифференциальной форме}
При прохождении тока по проводнику совершается работа по перенесению заряда q между точками с разностью потенциалов \emph{U}. Эта работа 
затрачивается на нагревание проводника
\begin{equation}\label{work}
 A = qU = IUt = I^2Rt = \frac{U^2}{R}t = Q
\end{equation}
В Международной системе единиц работа \emph{A} и количество тепла \emph{Q} измеряются в джоулях. Поэтому формула \ref{work} для работы 
тока одновременно выражает закон Джоуля-Ленца о тепловом действии тока. 

Заметим, что при \emph{последовательном} соединении сила тока \emph{I} во всех сечениях одинакова и поэтому, согласно формуле $Q = I^2Rt$,
количество тепла , выделяющееся на каком-либо участке цепи, пропорционально его сопротивлению $R$.

При \emph{параллельном} соединении во всех ветвях имеется общее напряжение U и количество тепла
\begin{equation}
 Q = \frac{U^2}{R}t, \nonumber
\end{equation}
выделяющееся в какой-либо ветви, обратно пропорционально ее сопротивлению, т.е. пропорционально ее проводимости.

Подсчитаем количество тепла, выделяемое за единицу времени в единице объема $V$ проводника, т.е. найдём 
\begin{equation}
 Q_1 = \frac{Q}{tV} = \frac{Q}{tSL}. \nonumber
\end{equation}
Используя соотношения \ref{rls}, \ref{Ell} и \ref{work}, получаем:
\begin{equation}\label{warm}
 Q_1 = \frac{U^2}{RSL} = \frac{E^2L^2}{\frac{1}{\gamma}\frac{L}{S}SL} = \gamma E^2.
\end{equation}
\emph{Колличество тепла, выделяемое в единице объема проводника за единицу времени, пропорционально квадрату напряженности электрического 
поля созданного в проводнике.}

Соотношение \ref{warm} называется законом Джоуля-Ленца в дифференциальной форме, так как определяет количество тепла, выделяемое в 1 сек
в единице объема.

В общем случае количнство тепла, выделяемое в 1 сек во всем объеме,
\begin{equation}\label{warmVol}
 Q_V = \int\limits_{(V)}Q_1\mathbf{d}V
\end{equation}
Рассматривая формулы \ref{vecj} и \ref{warm}, замечаем, какую важную роль играет величина напряженности электрического поля $E$, созданного
источником в проводнике. Величина $E$ определяет плотность тока в цепи и количество выделяемого током тепла.
\chapter{Лекция 14}
\section*{Замкнутая цепь с источником тока}
Рассмотрим замкнутую цепь электрического тока, составленную из источника тока и внешнего сопротивления $R$ \ref{img1}. Во внешней части цепи
заряды движутся под действием электростатических сил от большего потенциала к меньшему (путь $1R2$). Для того чтобы заставить заряды
двигаться внутри источника тока против направления электростатического поля (от 2 к 1 через внутренней сопротивление $r$), необходимо наличие
сил неэлектрического происхождения, так называемых сторонних сил.

Величину, определяемую работой, которую совершают сторонние силы при перемещении единичного положительного заряда по всей замкнутой цепи,
называют электродвижущей силой (э. д. с.) $\mathcal{E}$ в этой цепи (не путать с напряженностью электрического поля $\mathbf{E}$ и ее 
числовым значением $E$).
\begin{equation}\label{eds}
 \mathcal{E} = \frac{A_\text{ст}}{q_0}
\end{equation}
% here img1
Этой работой измеряется электростатическая энергия, которая получается в источнике тока за счет иных, неэлектрических форм энергии.

Из определения ЭДС видно, что это величина, аналогичная разности потенциалов (напряжению), имещая ту же размерность и измеряемая в тех же
единицах - вольтах.

Падение напряжения во внешней части цепи $U_1$ измеряется работой, совершаемой при перемещении единичного положительного заряда через 
внешнее сопротивление $R$
\begin{equation}\label{u1}
 U_1 = \frac{A_1}{q_0}
\end{equation}
При прохождении тока по всей замкнутой цепи внутри источника также затрачивается энергия на продвижение зарядов и имеет место внутреннее
падение напряжения $U_2$ эта энергия превращается в джоулево тепло внутри источника.
\begin{equation}\label{u2}
 U_2 = \frac{A_2}{q_0}
\end{equation}
Учитывая, что все три величины - $\mathcal{E}, U_1, \text{и} U_2$ измеряются работой по перемещению единичного положительного заряда и 
применяя закон сохранения энергии, получим
\begin{equation}\label{Aex}
 A_\text{ст} = A_1 + A_2
\end{equation}
\begin{equation}\label{Aex/q0}
 \frac{A_\text{ст}}{q_0} = \frac{A_1}{q_0} + \frac{A_2}{q_0} \text{ и } \mathcal{E} = U_1 + U_2.
\end{equation}
Эдс равна сумме падений напряжения во внешней и внутренней частях замкнутой цепи.

Устройства, в которых сторонние силы совершают работы и происходит превращение какого-либо вида энергии в электрическую, называются генераторами,
или источниками эдс (реже - источниками тока)ю

Подавляющая часть электрической энергии совершается за счет совершения механической работы в машинах, где используется явление электромагнитной
индукции (генераторы на стационарных и передвижных электростанциях). В термоэлементах (термопарах) происходит превращение тепловой энергии 
в электрическую. В гальванических элементах и аккумуляторах химическая энергия преобразуется в электрическую. Некоторые виды фотоэлементов
позволяют получать эдс за счет лучистой энергии.

Таким образом, следует различать электромеханические, тепловые, химические и оптические генераторы, или источники эдс. В настоящее время
разрабатываются новые генераторы - магнитогидродинаические, где тепловая энергия нагретого ионизированного газа или дыма непосредственно 
превращается в электрическую.

Получение различных радиоактивных изотопов позволило создать маломощные генераторы длительного действия; в них эдс получается за счет непрерывного
выбрасывания электронов радиоактивными ядрами.
\section{Закон Ома для всей цепи}
Используя закон Ома для участка цепи, можно заменить в формуле \ref{Aex/q0} $U_1$ и $U_2$ следующими выражениями:
\begin{equation}
 U_1 = IR,\text{ }U_2 = Ir.\nonumber
\end{equation}
Тогда соотношение \ref{Aex/q0} превратится в закон Ома для всей цепи: 
\begin{equation}\label{full_Ohm}
 \frac{\mathcal{E}}{I} = R + r
\end{equation}
Отношение эдс, действующей в замкнутой цепи, к силе тока есть величина постоянная, равная полному сопротивлению цепи или 
\begin{equation}\label{full_I}
 I = \frac{\mathcal{E}}{R + r}.
\end{equation}
Максимальный ток $I_\text{к.з.}$, как следует из закона Ома, получится если внешнее сопротивление R = 0, т.е. имеет место короткое замыкание
\begin{equation}\label{I_shcut}
 I_\text{к.з.} = \frac{\mathcal{E}}{r}
\end{equation}
Записав формулу \ref{full_Ohm} в виде
\begin{equation}
 \mathcal{E} = U_1 + Ir\nonumber
\end{equation}
можно определить эдс как величину, численно равную напряжению на зажимах источника при $I = 0$, т.е. при разомкнутой внешней цепи.
Этим пользуются для практического измерения эдс. Точное измерение эдс производится компенсационным способом (известным из лабораторной 
работы), при котором ток от источника не потребляется. Приближенно эдс измеряется вольтметром с большим сопротивлением.
\section{Полная и полезная мощноть. КПД источника}
Мощность, выделяемая во внешней цепи (нагрузке $R$), называется полезной
\begin{equation}
 P_1 = U_1I = I^2R\nonumber
\end{equation}
Внутри источника тока расходуется мощность 
\begin{equation}
 P_2 = I^2r\nonumber
\end{equation}
Полная мощность замкнутой цепи
\begin{equation}\label{full_P}
 P_n = I^2R + I^2r = IU_1 + IU_2 = I\mathcal{E}.
\end{equation}
Найдем условие, при котором полезная мощность будет максимальной. Для этого выразим полезную мощность как 
\begin{equation}
 P_1 = P_n - P_2 = I\mathcal{E} - I^2r\nonumber
\end{equation}
и приравняем нулю производную
\begin{equation}
 \frac{\mathbf{d}P_1}{\mathbf{d}I} = \mathcal{E} - 2Ir = 0,\nonumber
\end{equation}
откуда
\begin{equation}\label{profit_I}
 I = \frac{\mathcal{E}}{2r}.
\end{equation}
Сравнивая эту формулу с законом Ома \ref{full_I}, видим, что максимальна мощность во внешней цепи выделяется при равенстве внешнего и 
внутреннего сопротивления:
\begin{equation}\label{exeqin}
 R = r.
\end{equation}
Коэффициентом полезного действия $\eta$ источника эдс называется отношение полезной мощности к полной
\begin{equation}\label{nu}
 \eta = \frac{P_1}{P_n} = \frac{I^2R}{I^2(R + r)} = \frac{U_1}{\mathcal{E}} = \frac{R}{R + r}
\end{equation}
Наибольший кпд, равный единице, получится при равенстве нулю внутреннего сопротивления источника. Исходя из этого стараются делать внутреннее
сопротивление источника эдс (обмоток механического генератора, электролита и пластин аккумулятора) минимальным.
\section{Правила Кирхгофа}
Для расчета сложных электрических цепей удобно применять правила, установленные немецким физиком Кирхгофом. Первое из них относится к точке
разветвтления - узлу \ref{img2} : \emph{алгебраическая сумма сил токов, сходящихся в узле, равна нулю}.
\begin{equation}\label{kirchgoff1}
 \sum_1^n I_k = 0
\end{equation}
Приходящие к узлу токи считаются положительными, а уходящие - отрицательными. Применив первое правило Кирхгофа к узлу, изображенному на рис.25,
получим следующее выражение:
\begin{equation}
 I_1 - I_2 + I_3 - I_4 - I_5 = 0\nonumber
\end{equation}
По существу первое правило Кирхгофа выражает тот факт, что при установившемся токе заряды не могут скапливаться в узлах.

Второе правило Кирхгофа является обобщением закона Ома для всей цепи на случай наличия нескольких связанных контуров. Согласно этому правилу
\emph{при обходе замкнутого контура алгебраическая сумма произведений токов на соответствующие сопротивления (сумма падения напряжений) равна
алгебраической сумме э.д.с.}, т.е. при полном обходе замкнутого контура сумма подъемов и падений потенциала равна нулю.
\begin{equation}\label{kirchgoff2}
 \sum_1^n I_kR_k = \sum_1^m\mathcal{E}_i.
\end{equation}
При использовании правил Кирхгофа необходимо иметь в виду следующее:
\begin{enumerate}
 \item предварительно на схеме указывается предположительное направление токов;
 \item направление обхода контуров выбирается произвольно и сохраняется неизменным при решении задач;
 \item если направление данного тока совпадает с направлением обхода контура, то падение напряжения $IR$ считается положительным;
 \item э.д.с. считается положительной, если обход внутри данного источника совершается от отрицательного полюса к положительному;
 \item если после решения уравнений у какого-либо тока получается знак минус, то это означает, что его истинное направление противоположно выбранному;
 \item общее число независимых уравнений, составленных на основании обоих правил Кирхгофа, должно равняться числу токов в данной схеме
\end{enumerate}

Применим правила Кирхгофа к схеме, изображенной на \ref{img3}.

\textbf{Первое правило.} Его следует применять в узле $A$, или в узле $B$. Для узла $B$ имеем:
\begin{equation}\label{kirex1}
 I_1 + I_2 - I_3 = 0.
\end{equation}
Уравнение для узла $A$ идентично.

\textbf{Второе правило.} Выберем направление обхода, например, по часовой стрелке:
\begin{equation}\label{kirex2}
 \begin{cases}
    I_1R_1 + I_1r_1 - I_2r_2 - I_2R_2 = \mathcal{E}_1 - \mathcal{E}_2 \;\;\;\;\;\;\;(\text{Контур} A\mathcal{E}_1B\mathcal{E}_2A)\\
    I_2R_2 + I_2r_2 + I_3R_3 + I_3r_3 = \mathcal{E}_2 + \mathcal{E}_3 \;\;\;\;\;\;\;(\text{Контур} A\mathcal{E}_2B\mathcal{E}_3A)
 \end{cases}
\end{equation}
%img3 
Решая совместно \ref{kirex1} и \ref{kirex2}, найдем токи $I_1$, $I_2$ и $I_3$ (сопротивления и э.д.с. обычно известны).

Если бы мы составили третье уравнение  второму правилу Кирхгофа, а именно, обошли контур $A\mathcal{E}_1BR_3A$, то полученное уравнение
явилось бы следствием первых двух.
\chapter{Лекция 15}
\section*{Основные положения корпускулярной теории проводимости}
Как было сказано, током проводимости называют ток, созданный направленным движением электрически заряженных частиц - корпускул - в твердых телах,
 жидкостях или газах.
 
Рассмотрим проводник длиной $L$ с площадью поперечного сечения $S$ (\ref{img4}). Концентрацию носителей тока, т.е. число заряженных частиц 
в единице объема обозначим $n$, а заряд каждого носителя $q_0$. Если приложить к проводнику напряжение $U$, то под действием созданного в 
проводнике электрического поля на быстрое, хаотическое тепловое движение носителей наложится медленное, направленное перемещение их вдоль
поля со средней скоростью $v$. Если за время $t$ облак заряженных частиц переместилось на расстояние $L$, то очевидно $v = \frac{L}{t}$. 
За время $t$ через сечение проводника пройдут все заряды, находящиеся в части объема проводника $S\cdot L = V$. Поэтому учитывая, что концентрация
%img4
носителей n, а заряд каждого из них $q_0$, общая величина заряда $q$, перенесенного через сечение проводника за время $t$, выразим произведением
\begin{equation}\label{full_q}
 q = q_0nSL.
\end{equation}
Отсюда плотность тока
\begin{equation}\label{density}
 j = \frac{I}{S} = \frac{q}{tS} = \frac{q_0nSL}{tS} = q_0nv
\end{equation}
[см. формулы \ref{igreat}, \ref{ji}, \ref{full_q}].

Таким образом, \emph{плотность тока проводимости пропорциональна заряду каждого носителя, концентрации и средней скорости направленного движения 
носителей тока.} Как было сказано, плотность тока - векторная величина и по направлению совпадает с вектором скорости $\mathbf{v}$.

При наличии различных заряженных частиц - электронов, ионов, <<дырок>> - общая плотность тока
\begin{equation}\label{job}
 j = \sum_1^mj_i,
\end{equation}
где $j_i$ - плотность тока, создаваемого каждым из $m$ носителей.

В простейшем случае двух носиетелей, например положительных и отрицательных ионов, или <<дырок>> и электронов
\begin{equation*}
 j = j_+ + j_-
\end{equation*}
или
\begin{equation}\label{j154}
  j = q_{0+}n_+v_+ + q_{0-}n_-v_-
\end{equation}
В формуле (\ref{j154}) $q_{0+}$ и $q_{0-}$ - заряды положительных и отрицательных носителей, $n_+$ и $n_-$ - концетрация их, $v_+$ и $v_-$ 
- средние скорости.
\section{Подвижность носителей}
На заряженную частицу, участвующую в образовании тока проводимости, действует со стороны электрического поля, напряженность которого $E$, сила
\begin{equation}\label{155}
 F_\text{эл} = q_0E,
\end{equation}
ускоряющая эту частицу. При движении каждый носитель тока взаимодействует путем столкновения с другими частицами среды, передавая им некоторый 
импульс. В результате такого процесса среда тормозит движение носителя. Действующая при этом сила трения (сопротивления)
\begin{equation}\label{156}
 F_\text{тр} = kv
\end{equation}
предположительно может считаться пропорциональной средней скорости частицы. Например, при движении ионов в жидкости картина уподобляется движению 
шарика в вязкой среде, и сила трения определяется известной из молекулярной физики формулой Стокса
\begin{equation}
 F_\text{тр} = 6\pi \eta r v, 
\end{equation}
где $\eta$ - коэффициент вязкости среды, а r - радиус движущегося шарика.
Сравнивая формулу Стокса с выражением (\ref{156}), видим, что $k = 6\pi \eta r$, т.е. коэффициент трения $k$ зависит от вязкости (а следовательно, 
от температуры) и от размеров частицы.

При установившемся движении 
\begin{equation*}
 F_\text{эл} = F_\text{тр}.
\end{equation*}
Используя (\ref{155}) и (\ref{156}), получим 
\begin{equation}\label{157}
 q_0E = kv
\end{equation}
Величина $u$, численно равная средней скорости, с которой заряженная частица движется в данной среде под действием электрического поля с 
напряженностью, равной единице, называется подвижностью тока в данной среде 
\begin{equation}\label{158}
 u = \frac{v}{E} = \frac{q_0}{k}
\end{equation}
Используя определение подвижности (\ref{158}) и формулу (\ref{j154}), получим следующую формулу для плотности тока при наличии двух носителей
противоположных знаков:
\begin{equation}\label{159}
 j = q_{0+}n_+v_+ + q_{0-}n_-v_- = (q_{0+}n_+u_+ + q_{0-}n_-u_-)E.
\end{equation}
Сравнивая формулу (\ref{159}) для плотности тока с законом Ома в дифференциальной форме $j = \gamma E$, видим, что роль электропроводности играет
коэффициент при величине $E$
\begin{equation}\label{1510}
 \gamma = (q_{0+}n_+u_+ + q_{0-}n_-u_-).
\end{equation}
В СИ единицей подвижности является $1\frac{\text{м}}{\text{сек}}:1\frac{\text{в}}{\text{м}} = 1\frac{\text{м}^2}{\text{в}\cdot\text{сек}}$.
\section{Опытные предпосылки классической электронной теории металов}
В 1901 г. Рике доказал на опытах, что прохождение тока через металл не связано с переносом атомов металла. Через три цилиндрических проводника
, изготовленных из различных металлов и плотно прижатых друг к другу хорошо отшлифованными основаниями, пропускался постоянный ток. Опыт продолжался
свыше года, после чего цилиндры разобрали и проанализировали. Вес их не изменился, а химический состав в прилегавших областях изменился не больше, чем 
при обычной диффузии атомов в твердых телах. Этот опыт показывает, что атомы (ионы) металла не участвуют в создании тока.

Доказательством того, что носителями тока в металле являются именно электроны, служат опыты по обнаружению инерциального движения электронов.
Стюарт и Толмен (1916 г.) вращали проволочную катушку, имевшую длину провода $L$, с большой линейной скоростью $v$; свободные электроны вращались
вместе с катушкой. Затем катушку резко тормозили; электроны по инерции продолжали двигаться в прежнем направлении, причем их упорядоченное движение
делалось хаотическим, а кинетическая энергия превращалась в джоулево тепло
\begin{equation}\label{1511}
 -\mathbf{d}W_\text{к} = \mathbf{d}Q.
\end{equation}
Изменение кинетической энергии одного электрона
\begin{equation}\label{1512}
 \mathbf{d}(\frac{mv^2}{2}) = mv\mathbf{d}v,
\end{equation}
а их число в проволоке катушки $n\cdot S\cdot L$. Поэтому 
\begin{equation}\label{1513}
 dW_\text{к} = nSLmv\mathbf{d}v.
\end{equation}
Вспомнив, что $I = j \cdot S = envS$, заменим в выражении (\ref{1513}) произведение $nvS$ значением $\frac{I}{e}$, где $e$ - заряд электрона.

Используя равенства (\ref{1511}), (\ref{1513}) и закон Джоуля-Ленца, получаем:
\begin{equation}\label{1514}
 \mathbf{d}W_\text{k} = -I\frac{m}{e}L\mathbf{d}v = \mathbf{d}Q = I^2R\mathbf{d}t.
\end{equation}
Сокращаем на $I$ и заменяем на $I\cdot \mathbf{d}t = \mathbf{d}q$
\begin{equation}\label{1515}
 -\frac{m}{e}L\mathbf{d}v = R\mathbf{d}q.
\end{equation}
Интегрируем выражение (\ref{1515}), учитывая что скорость электронов изменяется от $v$ до $0$ (при торможении) и при этом по цепи протекает заряд
$q$:
\begin{equation*}
 -\frac{m}{e}L\int_v^0\mathbf{d}v = R\int_0^q\mathbf{d}q.
\end{equation*}
Отсюда
\begin{equation}\label{1516}
 \frac{m}{e}Lv = Rq \;\text{и}\;\frac{e}{m}=\frac{Lv}{Rq}.
\end{equation}
Длина проволоки катушки L и начальная скорость её вращения $v$ известны. Замыкая катушку на баллистический гальванометр (об этом приборе см. лекцию 40),
по его отбросу измеряют заряд $q$, протекший по цепи. Сопротивление катушки и гальванометра также известно. Таким образом, из опытных данных
может быть найдено отношение $\frac{e}{m}$, которое достаточно точно совпадает с числом, определенным из других опытов для свободных электронов.

В начале XX века физики Лоренц и Друде разработали классическую электронную теорию металлов. Атомы, образующие кристаллическую решетку металла, 
находятся на близких расстояниях и между ними действуют значительные электрические силы. Вследствие этого, валентные электроны отрываются от своих
атомов, оставляя в узлах решетки положительные ионы. Эти <<свободные>> электроны двигаются хаотически, взаимодействуя друг с другом и с ионами
решетки. Хотя силы этого взаимодействия велики, но средняя сила, действующая на данный свободный электрон со стороны остальных электронов и 
ионов, близка к нулю. Поэтому электроны, оторванные от атомов, считаются свободными. Их взаимодействие с ионами и другими электронами сводится к столкновениям.
При этом частицы обмениваются импульсом, энергией.

Лоренц и Друде предположили, что свободные электроны в металле представляют собой <<электронный газ>>, подчиняющийся законам, установленным
для молекул идеального газа; скорости свободных электронов распределяются согласно статистическому закону Максвелла-Больцмана (см. раздел 
<<Молекулярная физика>>).

По этой теории средняя кинетическая энергия свободных электронов определяется формулой энергии молекул газа
\begin{equation}\label{1517}
 \frac{mu^2_T}{2} = \frac{3}{2}kT.
\end{equation}
Впоследствии оказалось, что теория Друде и Лоренца не лишена принципиальных недостатков, так как энергия и скорости электронов подчиняются 
другому, более сложному закону (см. лекцию 17). Тем не менее классическая электронная теория сыграла известную положительную роль, позволив 
объяснить такие важные опытные законы, как закон Ома и Джоуля-Ленца.
\chapter{Электронная теория проводимости металлов}
Рассматривая согласно классической электронной теории металлов свободные электроны как электронный газ, можно вывести законы Ома и Джоуля-Ленца.

Рассмотрим объем проводника длиной $L$ и сечением $S$ (рис. \ref{img5}). Свободные электроны участвуют в тепловом движении, имея средние квадратичные
скорости
\begin{equation}\label{161}
 u_T = \sqrt{\frac{3kT}{m}},
\end{equation}
определяемые по формуле для скорости молекул газа. Подстановкой в формулу (\ref{161}) массы электрона $m = 9,1\cdot 10^{-31}\text{кг}$ и 
постоянной Больцмана $k = 1,38 \cdot 10^{-23} \emph{дж/град}$, получим для комнатной температуры $t = 17^\circ C$ 
или $T = 290^\circ K$ величину средней тепловой скорости электронов в металле порядка $100 \emph{км/сек}$.

Если на электроны подействует электрическое поле, то на их хаотическое тепловое движение наложится упорядоченное движение со средней скоростью
$v$, значительно меньшей тепловой скорости.

Действительно, при плотности тока в проводнике, равной, например, $5\emph{а/мм}^2 = 5\cdot 10^6 a/\text{м}^2$ и при концентрации свободных
электронов 
\begin{equation*}
 n \approx 10^{23} \cdot 1/\text{см}^3 = 10^{29} \cdot 1/\text{м}^3,
\end{equation*}
получим из известной формулы
\begin{equation*}
 j = nq_0v,
\end{equation*}
где $q_0$ равно заряду электрона $e = 1,6 \cdot 10^{-19}\text{к}$, 
\begin{equation*}
 v = \frac{j}{ne} = \frac{5 \cdot 10^6}{10^29\cdot 1,6 \cdot 10^{-19}} = 3 \cdot 10^{-4} \frac{\text{м}}{\text{сек}} = 0,3 \frac{\text{мм}{\text{сек}}.
\end{equation*}
Это упорядоченное движение электронов и образует ток в металле.
\section{Вывод закона Ома из электронной теории}
Рассмотрим движение свободного электрона в металле. После упругого удара об ион 1 (\ref{img6}) электрон отскочит от него, имея тепловую скорость
$u_T$, и полетит к иону 2.

За время свободного полета $\tau$ между двумя столкновениями с ионами на электрон, имеющий заряд $e$, со стороны электрического поля будет
действовать постоянная сила $F$.
%img 6
\begin{equation}\label{162}
 F = eE
\end{equation}
Под действием этой силы электрон будет двигаться равноускоренно и к моменту столкновения с ионом 2 скорость направленного движения электрона
станет равной $v_\tau$. Заметим, что после столкновения электронов с ионами средняя начальная скорость является тепловой ($u_T$). Это значит, 
что в результате столкновения электрон передал иону дополнительную кинетическую энергию, полученную в поле. Поэтому средняя начальная скорость
упорядоченного движения электрона $v_0 = 0$.

Среднее расстояние между ионами, с которыми электрон испытывает столкновения, является средней длиной свободного пробега.
\begin{equation}\label{163}
 \lambda = u_t\tau
\end{equation}
Хотя скорость электрона изменяется во время свободного пробега, как было показано, $v \ll u_T$ и при определении времени движения электрона
между двумя столкновениями считаем его скорость постоянной и равной $u_T$.

Ускорение $a$, которое приобретает электрон под действием силы поля,
\begin{equation*}
 a = \frac{F}{m} = \frac{eE}{m},
\end{equation*}
а скорость 
\begin{equation}\label{164}
 v_T = a_T = \frac{eE}{m}\tau = \frac{eE}{m} \cdot \frac{\lambda}{u_T}.
\end{equation}
Средняя скорость

\end{document}
